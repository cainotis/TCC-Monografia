%!TeX root=../tese.tex
%("dica" para o editor de texto: este arquivo é parte de um documento maior)
% para saber mais: https://tex.stackexchange.com/q/78101

%% ------------------------------------------------------------------------- %%

\chapter{Conclusão}
\label{cap:conclusao}

Em síntese, para obter resultados mais satisfatórios se faz necessário um maior desenvolvimento do projeto a fim de remover as insuficiências do modelo. Dessa maneira, é fundamental a implementação de novos métodos de otimização sejam capazes de acelerar o processo de aprendizado de forma que se torne possível uma mais rápida avaliação da evolução do treinamento da rede neural e, por conseguinte, a reavaliação dos critérios de pontuação. Por último, também é de suma importância um estudo mais aprofundado no que faz um bom sistema de pontuação para aprendizado por reforço, em virtude de estabelecer bons parâmetros de avaliação da evolução da rede além de permitir a localização com uma maior exatidão da causa do comportamento inesperado aprendido pelo robô.
