%!TeX root=../tese.tex
%("dica" para o editor de texto: este arquivo é parte de um documento maior)
% para saber mais: https://tex.stackexchange.com/q/78101

%% ------------------------------------------------------------------------- %%

% "\chapter" cria um capítulo com número e o coloca no sumário; "\chapter*"
% cria um capítulo sem número e não o coloca no sumário. A introdução não
% deve ser numerada, mas deve aparecer no sumário. Por conta disso, este
% modelo define o comando "\unnumberedchapter".
\unnumberedchapter{Introdução}
\label{cap:introducao}

\enlargethispage{.5\baselineskip}

\emph{“Can machines think?”} (máquinas conseguem pensar?), uma simples pergunta proposta por Alan M. Turing\index{Alan M. Turing} em seu \emph{paper Computing Machinery and Intelligence}~\citep{turing1950computing} mas com repercussões que iriam alterar o rumo da humanidade para sempre. Com a revolução industrial, os rápidos avanços da ciência permitiram que ideias antes presas na ficção e nos mitos entrassem nas vidas cotidianas das pessoas e, com o passar do tempo, criou-se no subconsciente humano a ideia de um dia ser possível construir objetos capazes de pensar. Essa ideia começa a tomar forma concreta com a publicação do livro \emph{The Wonderful Wizard of Oz}~\citep{Baum_Denslow_1996}, no qual um homem feito de lata é capaz de se mover e pensar como uma pessoa normal, e alguns anos depois com estreia a peça \emph{“R.U.R.: Robôs Universais de Rossum”} do escritor tcheco Karel Čapek\index{Karel Čapek} o termo ‘robô’ é cunhado, o qual permite que a imaginação humana junte às mais diversas concepções de inteligência artificial e criação de máquinas, antes referidas apenas como autômatos.

Essa ideia influenciou autores de ficção científica por todo o mundo, porém apenas em 1950 alguém para para refletir se é realmente possível se construir uma máquina capaz de pensar. Em seu trabalho, o matemático britânico Alan M. Turing propõe um teste, ao qual ele se refere como “The Imitation Game”~\citep{turing1950computing} ou em portugues “O jogo da imitação”, no qual um indivíduo, referido como interrogador, deve identificar através de uma série de perguntas e respostas algum “rótulo” de dois outros indivíduos enquanto eles tentam fazer com que o interrogador erre na identificação. Após a definição do jogo, ele propõe o experimento mental em que uma máquina participa do jogo no lugar de um dos interrogados e questiona se o interrogador irá errar quem é o humano e quem é a máquina na mesma proporção que ele é algo como quem é o homem e quem é a mulher. Essa reformulação transforma a pergunta na questão de se é possível fazer com que uma máquina(computador) e com um dado conjunto de instruções (programa) seja capaz de imitar de forma indistinguível o comportamento humano.

Essa nova pergunta despertou a curiosidade de inúmeros cientistas pelo mundo e levou, não somente à criação de programas de processamento de linguagem natural, como ELIZA ~\citep{weizenbaum1966eliza}, mas também, inúmeras pesquisas deram origem à inteligência artificial e, mais importante para esse trabalho, ao aprendizado de máquina. O aprendizado de máquina, diferente das demais inteligências artificiais, se caracteriza pela criação de sistemas computacionais capazes de aprender e evoluir ao usar de modelos estatísticos para convergir para uma aproximação do modelo matemático capaz de resolver o problema dado. Contudo tais capacidades vem através de custosos processos de treinamento além uma grande quantidade de informações sobre o problema a ser resolvido. Assim, seus primeiros sucessos ficaram presos ao laboratório pois seu custos operacionais inviabilizam o uso prático, como foi o caso do trabalho Leonard Uhr e Charles Vossler ~\citep{Uhr_Vossler_1961}, um dos primeiros sucessos em desenvolver um programa capaz de reconhecer letras a partir de imagens, e foi necessário mais alguns anos de evoluções tecnológicas para que seus resultados fossem aplicados no cotidiano.

O aumento do poder computacional torna possível o desenvolvimento das metodologias de aprendizado por reforço que se baseiam em treinar o agente em um ambiente simulado. Em suma, nessas técnicas o agente aprende a maximizar sua pontuação por processos de tentativa e erro, os quais permitem uma exploração sistemática do ambiente simulado o que leva o agente a descobrir a solução do problema uma vez sua pontuação em cada iteração de treino é baseada em quão bem é o seu desempenho em resolver o problema. Contudo, essas técnicas demandam um largo poder computacional, visto que o espaço de ações possíveis para cada estado é infinitamente maior que o conjunto de ações corretas. Dessa forma, acelerar o treinamento consiste principalmente em desenvolver maneiras de eliminar mais rapidamente caminhos que não são capazes de melhorar o resultado atual, sem que haja perda em comparação com fazer uma exploração completa do ambiente e aumentar a paralelização, de forma geral, do processo de treinamento para maximizar a utilização dos recursos computacionais disponíveis.

Neste trabalho é estudado o uso de um modelo de aprendizado por reforço para a criação e desenvolvimento de veículos completamente autônomos em ambiente urbano. Para isso o simulador utilizado para definir e manipular o ambiente é o Duckietown o qual é utilizado na disciplina MAC0318 - Introdução à Programação de Robôs Móveis. Com o Duckietown é simulado um veículo simples que deve ser capaz de percorrer um circuito na cidade dos patos, que será discutido no capítulo \ref{cap:duckietown}. Além disso, o método de aprendizagem por reforço utilizado é o Deep Q-Network (DQN) que combina redes neurais com método Q-learning que será discutido no capítulo \ref{cap:dqn}. No capítulo \ref{cap:metodologia}, é apresentada a metodologia que foi usada para desenvolver o projeto. Por fim, no capítulo \ref{cap:conclusao}, é discutido o resultado obtido e as conclusões do trabalho.