%!TeX root=../tese.tex
%("dica" para o editor de texto: este arquivo é parte de um documento maior)
% para saber mais: https://tex.stackexchange.com/q/78101

% As palavras-chave são obrigatórias, em português e em inglês, e devem ser
% definidas antes do resumo/abstract. Acrescente quantas forem necessárias.
\palavrachave{Inteligência Artificial}
\palavrachave{Duckietown}
\palavrachave{Aprendizado de Máquina}

\keyword{Artificial Intelligence}
\keyword{Duckietown}
\keyword{Machine Learning}

% O resumo é obrigatório, em português e inglês. Estes comandos também
% geram automaticamente a referência para o próprio documento, conforme
% as normas sugeridas da USP.
\resumo{
Neste trabalho se estuda o uso de um modelo de aprendizado por reforço para a criação e desenvolvimento de veículos completamente autônomos em ambiente urbano. Para isso se utiliza o simulador Duckietown para definir o ambiente de treino, que consiste em um simples circuito no qual o veiculo deve percorrer sem cometer infrações. Para a aprendizagem é utilizado o Deep Q-Network (DQN), que combina redes neurais e deep learning com classico método de aprendizagem por reforço Q-learning. Com as punições e recompensas utilizadas para treinar o modelo é observado que nem sempre o veículo aprende o esperado, ao encontrar diferentes formas de trapacear o sistema de pontuação.
}

\abstract{
This work is the study of the use of a reinforced learning model for the creation and development of autonomous vehicles in the urban environment. It uses the Duckietown simulator to define the training environment, which is made of a simple circuit that the car has to run without making any infractions. To learn, the agent uses the Deep Q-Network algorithm, which is a combination of neural networks and deep learning with the reinforced learning method classic Q-learning. With the punishments and rewards used to train the model, it is observed that the vehicle does not always learn what is expected when it finds different ways to cheat the scoring system.
}
