%!TeX root=../tese.tex
%("dica" para o editor de texto: este arquivo é parte de um documento maior)
% para saber mais: https://tex.stackexchange.com/q/78101

%% ------------------------------------------------------------------------- %%

\chapter{Metodologia}
\label{cap:dqn}

Este estudo de caso se faz com a utilização do \emph{framework} TensorFlow 2.0, uma atualização do TensorFlow original lançada pelo Google\index{Google} em setembro de 2019 ~\citep{TensorFlow2-release}\index{TensorFlow}. A utilização de uma \emph{framework}, como o TensorFlow, simplifica os processos de criação, de treino e de execução do agente inteligente, o que, por um lado, permite um maior investimento energético em melhorar as definição do ambiente e em desenvolver as métricas utilizadas para recompensar ou punir as ações tomadas pelo agente durante o treino, contudo, por outro lado, passa o controle do treino e desenvolvimento do agente para a \emph{framework}, por conseguinte, os transforma em uma caixa preta cujo o funcionamento só é conhecido pelos desenvolvedores de dada \emph{framework}.

